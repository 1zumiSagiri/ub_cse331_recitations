\documentclass[t]{beamer}
\usepackage{graphicx}
\usepackage{hyperref} % hypertext link
\usepackage[backend=biber,style=numeric,sorting=none]{biblatex} % bibliography, install biber otherwise it will not work
\addbibresource{intro.bib} % bibliography file name *.bib


% set the theme
\usetheme{Boadilla}
\usefonttheme{professionalfonts}
\setbeamerfont{section in toc}{size=\normalsize}
\setbeamerfont{subsection in toc}{size=\small}
\setbeamerfont{section number projected}{size=\small}
\setbeamerfont{subsection number projected}{size=\small}
\setbeamerfont{footnote}{size=\tiny}


% show the table of contents before each subsection
% and highlight the current subsection
%
\AtBeginSubsection[]{
  \begin{frame}
    \frametitle{Overview}
    \tableofcontents[currentsection, currentsubsection]
  \end{frame}
}


% only show the table of contents before each section
% and highlight the current section
%
% \AtBeginSection[]{
%   \begin{frame}
%     \frametitle{Overview}
%     \tableofcontents[currentsection]
%   \end{frame}
% }


\begin{document}

    \author{Vincent Chan}
    \title{CSE331 Proof Walkthrough}
    \date{\today}


    % show the title page
    %
    \begin{frame}
        \titlepage
    \end{frame}


    % show the table of contents
    % \section{Course Overview}
    %
    % \section{Proofs}
    % \subsection{Why proofs?}
    % \subsection{Proof Idea \& Proof Detail}
    %
    % \section{Proof Techniques}
    % \subsection{Proof by Induction}
    % \subsection{Proof by Contradiction}
    % \subsection{Reduction}
    % \subsection{Proof by Counterexample}
    %
    % \section{Asymptotic Analysis}
    % \subsection{Big-O Notation}
    % \subsection{Big-$\Omega$ Notation}
    % \subsection{Big-$\Theta$ Notation}
    

    \begin{frame}
        \frametitle{Overview}
        \tableofcontents
    \end{frame}


    %%%%%%%%%%%%%%%%%%%%% BEGIN CONTENT %%%%%%%%%%%%%%%%%%%%%

    \section{Course Overview}
        \begin{frame}
            \frametitle{Course Overview}
            \begin{block}{\href{https://www-student.cse.buffalo.edu/~atri/cse331/fall24/policies/syllabus.html}{Course Website}}
                \begin{itemize}
                    \item Recitation notes for weekly homework.
                    \item Background Material
                    \item More resources
                \end{itemize}
            \end{block}
        \end{frame}

    \section{Proofs}
        \subsection{Why Proofs?}
            \begin{frame}
                \frametitle{Why Proofs?}
            \end{frame}

        \subsection{Proof Idea \& Proof Detail}
            \begin{frame}
                \frametitle{Proof Idea \& Proof Details}
                \begin{block}{Proof Idea}
                    The high-level idea of the proof.(English)
                \end{block}

                \begin{block}{Proof Details}
                    The detailed steps of the proof. (Pseudocode, etc.)
                \end{block}
            \end{frame}

    \section{Proof Techniques}
        \subsection{Proof by Induction}
            \begin{frame}
                \frametitle{Proof by Induction}
                \begin{block}{Principle of Mathematical Induction}
                    To prove that a statement $P(n)$ is true for all positive integers $n$, we need to show:
                    \begin{enumerate}
                        \item<1-> Base case: $P(1)$ is true.
                        \item<2-> Inductive step: If $P(k)$ is true, then $P(k+1)$ is true.
                    \end{enumerate}
                \end{block}
            \end{frame}

            \begin{frame}
                \frametitle{Proof by Induction}
                \begin{block}{Example}
                    Prove that $\sum_{i = 1}^{n} i = \frac{n(n+1)}{2}\ \forall n \in \mathbb{N}$.
                \end{block}
            \end{frame}

            \begin{frame}
                \frametitle{Proof by Induction}
                \begin{block}{Base case}
                    $P(1)$ is true because $1 = \frac{1(1+1)}{2}$.
                \end{block}

                \begin{block}{Inductive hypothesis}
                    Assume $P(k)$ is true, i.e., $\sum_{i = 1}^{k} i = \frac{k(k+1)}{2}$.
                \end{block}
            \end{frame}

            \begin{frame}
                \frametitle{Proof by Induction}
                \begin{block} {Inductive step}
                    We need to show that $P(k+1)$ is true, i.e., $\sum_{i = 1}^{k+1} i = \frac{(k+1)(k+2)}{2}$.

                    \bigskip
                    Base on the inductive hypothesis, we have:
                    \[\sum _{i = 1}^{k+1} i = \sum _{i = 1}^{k} i + (k+1) = \frac{k(k+1)}{2} + (k+1) = \frac{(k+1)(k+2)}{2}\]

                    \bigskip
                    Therefore, $P(k+1)$ is true. We proof by induction that $\sum_{i = 1}^{n} i = \frac{n(n+1)}{2}\ \forall n \in \mathbb{N}$.
                \end{block}
            \end{frame}

            \subsection{Proof by Contradiction}
                \begin{frame}
                    \frametitle{Proof by Contradiction}
                    Assume the statement is false, then derive a contradiction.
                \end{frame}

                \begin{frame}
                    \frametitle{Proof by Contradiction: Pigeonhole Principle}
                    \begin{block}{Pigeonhole Principle}
                        If $n+1$ pigeons are placed into $n$ pigeonholes, then at least one pigeonhole contains 2 pigeons.
                    \end{block}
                \end{frame}

            \begin{frame}
                \frametitle{Proof by Contradiction: Pigeonhole Principle \footfullcite{pigeonhole}}
                \begin{block}{Proof}
                    Assume that no pigeonhole contains 2 pigeons.

                    \bigskip
                    For notational convenience, let $h_i$ denote the number of pigeons in the $i$-th hole for every $i \in [n]$. For the sake of contradiction, 
                    let us assume that $h_i \leq 1$ for every $i \in [n]$. Now since each pigeon is assigned to some hole, 
                    the total number of pigeons is \[\sum_{i=1}^{n} h_i \leq \sum_{i=1}^{n} 1 = n\], where the inequality follows from our assumption that $h_i \leq 1$ for every $i \in [n]$. 
                    This implies that the total number of pigeons is at most $n$, which is a contradiction since there are $n+1$ pigeons. 
                    This implies that there exists at least one $i \in [n]$ s.t. $h_i \geq 2$, as desired.
                \end{block}
            \end{frame}

        \subsection{Reduction}
            \begin{frame}
                \frametitle{Reduction}
                To prove that problem $A$ can be solved by solving problem $B$, we reduce $B$ to $A$.

                \begin{block}{Example}
                    Example $\cdots$
                \end{block}
            \end{frame}


        \subsection{Proof by Counterexample}
            \begin{frame}
                \frametitle{Proof by Counterexample}
                To prove that a statement is false, provide a counterexample. Useful when dealing 
                with universal quantifiers (e.g., $\forall$).
            \end{frame}

            \begin{frame}
                \frametitle{Proof by Counterexample}
                \begin{block}{Example}
                    Prove that the statement ``There are no even prime numbers'' is false.
                \end{block}
                \begin{block}{Proof}
                    2 is an even prime number.
                \end{block}
            \end{frame}

    \section{Asymptotic Analysis}


        \subsection{Big-O Notation}
            \begin{frame}
                \frametitle{Big-O Notation}
                \begin{block}{Definition}
                    $f(n) = O(g(n))$ if $\exists c > 0$ and $n_0 > 0$ s.t. \[0 \leq f(n) \leq c \cdot g(n)\ \forall n \geq n_0\].
                \end{block}

                \begin{block}{Example}
                    $3n^2 + 5n + 7 = O(n^2)$.
                \end{block}
            \end{frame}

        \subsection{Big-$\Omega$ Notation}
            \begin{frame}
                \frametitle{Big-$\Omega$ Notation}
                \begin{block}{Definition}
                    $f(n) = \Omega(g(n))$ if $\exists c > 0$ and $n_0 > 0$ s.t. \[0 \leq c \cdot g(n) \leq f(n)\ \forall n \geq n_0\].
                \end{block}

                \begin{block}{Example}
                    $3n^2 + 5n + 7 = \Omega(n^2)$.
                \end{block}
            \end{frame}

        \subsection{Big-$\Theta$ Notation}
            \begin{frame}
                \frametitle{Big-$\Theta$ Notation}
                \begin{block}{Definition}
                    $f(n) = \Theta(g(n))$ if $\exists c_1 > 0$, $c_2 > 0$, and $n_0 > 0$ s.t. 
                    \[0 \leq c_1 \cdot g(n) \leq f(n) \leq c_2 \cdot g(n)\ \forall n \geq n_0\].
                \end{block}

                \begin{block}{Example}
                    $3n^2 + 5n + 7 = \Theta(n^2)$.
                \end{block}
            \end{frame}


    \section{Proof Idea \& Proof Detail}


        \begin{frame}
            \frametitle{Proof Idea \& Proof Details}
            \begin{block}{Proof Idea}
                The high-level idea of the proof.
            \end{block}

            \begin{block}{Proof Details}
                The detailed steps of the proof.
            \end{block}
        \end{frame}


    \begin{frame}
        \frametitle{Any Questions?}
        Post your questions on Piazza!

        \bigskip
        Also, try to come to office hours and ask questions.
    \end{frame}


    %%%%%%%%%%%%%%%%%%%%% END CONTENT %%%%%%%%%%%%%%%%%%%%%


    % show the references
    %
    \begin{frame}
        \frametitle{References}
        \printbibliography
    \end{frame}

\end{document}
