\documentclass[t]{beamer}
\usepackage{graphicx}
\usepackage{hyperref} % hypertext link
\usepackage{amsmath} % math symbols
\usepackage{xcolor}

% set the theme
\usetheme{Boadilla}
\usefonttheme{professionalfonts}
\setbeamerfont{section in toc}{size=\normalsize}
\setbeamerfont{subsection in toc}{size=\small}
\setbeamerfont{section number projected}{size=\small}
\setbeamerfont{subsection number projected}{size=\small}
\setbeamerfont{footnote}{size=\tiny}


% show the table of contents before each subsection
% and highlight the current subsection
%
\AtBeginSubsection[]{
  \begin{frame}
    \frametitle{Overview}
    \tableofcontents[currentsection, currentsubsection]
  \end{frame}
}

\begin{document}

    \author{Vincent Chan}
    \title{CSE331: $\mathcal{P}\ \&\ \mathcal{NP}$}
    \date{\today}


    % show the title page
    %
    \begin{frame}
        \titlepage
    \end{frame}

    %%%%%%%%%%%%%%%%%%%%% BEGIN CONTENT %%%%%%%%%%%%%%%%%%%%%

    \section{Problem Overview}
        \begin{frame}
            \frametitle{Problem Overview}
            \begin{block}{So far what we have learned}
                \begin{itemize}
                    \item<1-> Can be solved in polynomial time
                    \item<2-> polynomial time: $O(n^k)$ for some constant $k \geq 0$
                \end{itemize}
            \end{block}

            \onslide<3-> \textbf{class $\mathcal{P}$}: Problems that can be answered in polynomial time.
        \end{frame}

        \begin{frame}
            \frametitle{New (?) Problem}
            \onslide<1-> \textbf{class $\mathcal{NP}$}: Problems that given a solution, correctness can be \textit{determined} in polynomial time.\\
            \onslide<2-> \textbf{determined}: Given a solution, check if it is correct $\leftarrow$ \textcolor{red}{Boolean output}.

            \bigskip
            \onslide<3-> \textbf{Is shortest path problem $\mathcal{NP}$?}\\
            \onslide<4-> \textbf{Answer}: No, because the solution is not a Boolean output.

            \bigskip
            \onslide<5-> \textbf{Given a path $s \to t$, can we verify if it is the shortest path $s \to t$ in G?}\\
            \onslide<6-> \textbf{Answer}: Yes.
        \end{frame}

        \begin{frame}
            \frametitle{New (!) Problem}
            \onslide<1-> Hardest NP problems, all NP problems can be reduced to it.\\
            \onslide<2-> \textbf{class $\mathcal{NP}-complete$}:
            \begin{itemize}
                \item<3-> is in $\mathcal{NP}$.
                \item<4-> at least one problem in $\mathcal{NP}$-complete can be reduced to it.
            \end{itemize}

            \bigskip
            \onslide<5-> \textbf{Easy to verify, hard to solve}: no polynomial time algorithm to solve it ($O(n!)$, $O(2^n)$, etc.)\\
            \onslide<6-> \textbf{Example}: SAT problem (first NPC), Hamiltonian cycle/Traveling salesman problem (in sample midterm), dependency hell problem (package manager), etc.
        \end{frame}

        \begin{frame}
            \frametitle{Why we care about $\mathcal{NP}$?}
            \onslide<1-> Recall $\mathcal{NP}$ problems are easy to verify, hard to solve.\\
            \onslide<2-> $\mathcal{NP}$-complete problems are the hardest problems in $\mathcal{NP}$.\\

            \bigskip
            \onslide<3-> Cryptography 101: given a password, it is easy to verify if it is correct, but brute force to find the password is hard (RSA, ED25519).\\
            \onslide<4-> If we can solve one $\mathcal{NP}$-complete problem in polynomial time, we can solve all $\mathcal{NP}$-complete problems in polynomial time.\\
            \onslide<5-> If you forgot your password, the good news is that you can ask someone to find it in short time$\cdots$\\
        \end{frame}

        \begin{frame}
            \frametitle{P, NP, NPC, NP-hard$\cdots$}
            We believe that $\mathcal{P} \neq \mathcal{NP}$, but \textbf{not sure}.\\

            \bigskip
            \onslide<1-> \textbf{class $\mathcal{NP}-hard$}: Problems that are at least one problem in $\mathcal{NP}$-complete can be reduced to it, not necessarily in $\mathcal{NP}$.\\
            \onslide<2-> \textbf{class Undecidable}: Problems that cannot be solved by any algorithm, e.g.halting problem.\\
        \end{frame}

    \section{Reduction}
        \begin{frame}
            \frametitle{Y $\leq_P$ X}
            \onslide<1-> \textbf{Proof a problem Y is in $\mathcal{NP}$-complete}:\\
            \onslide<2-> \textbf{Idea}: If we can solve problem X in $O(\mathcal{N})$, we can solve problem Y in $O(\mathcal{N})$.\\
            \onslide<3-> \textbf{Reduction}: Given a problem Y, we can reduce it to another problem X.\\
            \begin{itemize}
                \item<4-> \textbf{Pre-process}: Given an instance of problem Y, we can transform it to an instance of problem X in polynomial time.
                \item<5-> \textbf{Solve}: Given an instance of problem X, we can solve it in polynomial time.
                \item<6-> \textbf{Post-process}: Given a solution of problem X, we can transform it to a solution of problem Y in polynomial time.
            \end{itemize}
            \bigskip
            \onslide<7-> In a vocal way, we can say that problem X is \textbf{at least as hard as} problem Y.\\
            \bigskip
            \onslide<8-> If problem X is in $\mathcal{NP}$-complete, then problem Y is also in $\mathcal{NP}$-complete.\\
        \end{frame}

    %%%%%%%%%%%%%%%%%%%%% END CONTENT %%%%%%%%%%%%%%%%%%%%%

\end{document}
